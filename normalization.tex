
\subsection{Preparing experimental data for use with ANIMO}
The measurements obtained from the digital imaging analysis of the PP96 plates were stored in a spreadsheet.
We will now illustrate the process through which this raw data was normalized and rescaled
to make it suitable for working with ANIMO.

A spreadsheet with all the data generated by applying the following steps is available in the Supplementary Materials.

\subsubsection{Background subtraction}
In order to estimate the background noise in the data, three wells in each plate
were kept empty (blank wells). 
A value of background noise was calculated for each spot as the average value of that spot in the blank wells.
The background noise values were then subtracted from the data of the corresponding spots in all the other wells.

\subsubsection{Normalization}
As both the reference and the normalization HSP60 spots yielded inconsistent results,
we decided to normalize the data grouping them by treatment condition and using one plate as reference.
Plate 3 (C20A4 cell line) was chosen as reference, as it shared conditions with both the other plates (donor cells).
Two normalization groups were then defined, separating the data for IL-1B treatment in plate 1 from
the data for treatments with Wnt-3a and Wnt-3a + IL-1B in plate 2.
The two groups were associated to the corresponding two groups in plate 3.

For each of the five considered spots (Akt, ERK, GSK-3B, JNK1, p38),
the average of the data was computed over all time points in each group. These values were then divided
by the average of the corresponding data in plate 3, obtaining the normalization factor.
In two cases (JNK1 and p38) the treatment with Wnt-3a was not considered 
for the computation of the normalization factor, as the response in that case was
considerably lower than with the other two treatments.

At the end of this process, each of the five spots in plate 1 and plate 2 had their own normalization factor, while
all factors for plate 3 were equal to 1. Each data point was then divided by the corresponding normalization factor.

\subsubsection{Data scaling}
As ANIMO models are semi-quantitative, it is not required to precisely estimate the molecular concentrations
in experimental data. For this reason, we rescaled all time series on a 0-100 interval,
dividing all the values by their maximum over a normalization group. For instance,
the scaled value for ERK in plate 2 is obtained dividing each normalized value by
the maximum normalized value of ERK over the two treatment conditions Wnt-3a and Wnt-3a + IL-1B in plate 2.

\subsubsection{Average series}
As the experiments were done in triplicate, three time series are available
for each treatment and spot. The time series we use as reference in our ANIMO models
contain average and standard deviation values over the triplicate values.