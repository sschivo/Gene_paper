
%FIRST EXAMPLE STYLE

\tikzset{titregris/.style =
{draw=gray, thick, fill=white, shading = exersicetitle, %
text=black!70, rectangle, rounded corners, right,minimum height=.7cm}}
\pgfdeclarehorizontalshading{exersicebackground}{100bp}
{color(0bp)=(green!20); color(100bp)=(black!5)}
\pgfdeclarehorizontalshading{exersicetitle}{100bp}
{color(0bp)=(green!20);color(100bp)=(blue!5!white)}
\makeatletter
\mdfdefinestyle{ANIMOstyle}{%
nobreak=true,
outerlinewidth=1em,outerlinecolor=white,%
leftmargin=-1em,rightmargin=-1em,%
middlelinewidth=1.2pt,roundcorner=5pt,linecolor=gray,
apptotikzsetting={\tikzset{mdfbackground/.append style ={%
shading = exersicebackground}}},
innertopmargin=1.2\baselineskip,
skipabove={\dimexpr0.5\baselineskip+\topskip\relax},
skipbelow={-1em},
needspace=3\baselineskip,
frametitlefont=\sffamily\bfseries,
singleextra={%
\node[titregris,xshift=0.5cm] at (P-|O) %
{~\mdf@frametitlefont{\small What is ANIMO?}\hbox{~}};},
}
\makeatother

\begin{mdframed}[style=ANIMOstyle]
\pretolerance=2000
\scriptsize
ANIMO (Analysis of Networks with Interactive MOdeling)
is a software tool that allows the biologists to virtually \emph{play} with abstract network models.
Such models are based on cause-and-effect relationships, such as ``A activates B'', and their
behavior is described in semi-quantitative terms. This approach requires less precise knowledge about
kinetic parameters, while still enabling the user to easily investigate the dynamics of a modeled network.

The idea at the base of ANIMO is to provide experts with a means to visually represent their knowledge on biological interactions.
Networks created in this way are then enriched with simplified kinetics, producing \emph{dynamic models}.
Thanks to the Cytoscape-based user interface, the visual representation of a pathway available in ANIMO
closely resembles the networks commonly found in textbooks.

The network topology, equipped with informations
on its dynamic behavior, is automatically translated to an underlying formal model based on Timed Automata.
The formal model is used \emph{behind the scenes} to simulate the evolution of the network.
In a few seconds of computation, ANIMO presents the user with a graph plotting the levels of activation
of all components in the network during the simulation period.
A slider under the graph lets the user further explore the evolution of the model:
moving the slider makes the coloration of the network instantly reflect the activity of
all components along the simulation interval.
\end{mdframed}



%SECOND EXAMPLE


\mdfdefinestyle{ANIMObox}{%
    nobreak=true,
    outerlinecolor = gray!80!black,
    outerlinewidth=2pt,
    roundcorner=20pt,
    innertopmargin=\baselineskip,
    innerbottommargin=\baselineskip,
    innerrightmargin=10pt,
    innerleftmargin=10pt,
    backgroundcolor = gray!20!white}

\begin{mdframed}[style=ANIMObox]
{\bf Analysis of Networks with Interactive MOdeling}
\pretolerance=2000

\small
ANIMO is a software tool that allows the biologists to virtually \emph{play} with abstract network models.
Such models are based on cause-and-effect relationships, such as ``A activates B'', and their
behavior is described in semi-quantitative terms. This approach requires less precise knowledge about
kinetic parameters, while still enabling the user to easily investigate the dynamics of a modeled network.

The idea at the base of ANIMO is to provide experts with a means to visually represent their knowledge on biological interactions.
Networks created in this way are then enriched with simplified kinetics, producing \emph{dynamic models}.
Thanks to the Cytoscape-based user interface, the visual representation of a pathway available in ANIMO
closely resembles the networks commonly found in textbooks.

The network topology, equipped with informations
on its dynamic behavior, is automatically translated to an underlying formal model based on Timed Automata.
The formal model is used \emph{behind the scenes} to simulate the evolution of the network.
In a few seconds of computation, ANIMO presents the user with a graph plotting the levels of activation
of all components in the network during the simulation period.
A slider under the graph lets the user further explore the evolution of the model:
moving the slider makes the coloration of the network instantly reflect the activity of
all components along the simulation interval.
\end{mdframed}
